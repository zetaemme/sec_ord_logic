\documentclass{beamer}

\usepackage{amssymb}

\usetheme{Boadilla}

\title{Second-Order Logic}
\subtitle{Final seminar for "Logic in Computer Science"}
\author{Mattia Zorzan}
\institute{University of Verona}
\date{\today}

\begin{document}
    \begin{frame}
        \titlepage
    \end{frame}

    \begin{frame}
        \frametitle{Outline}
        \tableofcontents
    \end{frame}

    \section{Introduction}
        \begin{frame}
            \frametitle{Introduction}
            \begin{itemize}
                \item First-order logic allows "iteration" over the \textit{elements} of a structure
                \item Happens thanks to \textbf{quantifiers}: $ \forall $, $ \exists $
                    \begin{itemize}
                        \item $ \forall x. \phi(x) \rightarrow $ "For each $ x $, $ x $ satisfies the formula $ \phi $"
                        \item $ \exists x. \phi(x) \rightarrow $ "There exists $ x $ s.t. the formula $ \phi $ is satisfied"
                    \end{itemize}
                \item Limiting since we may only need to range over \textit{subsets} or "\textit{combinations}" (e.g. \textit{Cartesian product})
            \end{itemize}
        \end{frame}

        \subsection{A recall on first-order logic}
            \begin{frame}
                \frametitle{A brief recall}
                \begin{itemize}
                    \item Second-order logic "extends" first-order logic
                    \item Since that, let's recall the basics of first-order logic
                    \item Two key parts:
                        \begin{itemize}
                            \item \textit{Syntax}: Which sequences constitute \textbf{well-formed} expressions
                            \item \textit{Semantics}: The \textbf{meaning} behind this expressions
                        \end{itemize}
                \end{itemize}
            \end{frame}

            \subsubsection{Syntax}
                \begin{frame}
                    \frametitle{Syntax - Introduction}
                    \begin{itemize}
                        \item Two base types:
                            \begin{itemize}
                                \item \textbf{Terms}: Represents \textit{objects}
                                \item \textbf{Formulas}: Represents \textit{predicates}
                            \end{itemize}
                        \item Both formed by \textit{symbol} concatenation
                        \item All symbols together form the \textbf{alphabet} of the language
                        \item Can divide symbols in two categories
                            \begin{itemize}
                                \item \textit{Logical} symbols
                                \item \textit{Non-logical} symbols
                            \end{itemize}
                    \end{itemize}
                \end{frame}

                \begin{frame}
                    \frametitle{Syntax - Logical symbols}
                    \begin{itemize}
                        \item \underline{Infinite} set of \textbf{variables}: $ x, y, z, \dots, x_0, x_1, \dots $ (Lowercase letters)
                        \item \textbf{Connectives}: $ \wedge, \vee, \Rightarrow, \neg $
                        \item \textbf{Quantifiers}: $ \forall, \exists $
                        \item \textbf{Equality} (or \textit{Identity}): $ = $
                        \item \textbf{Auxiliary symbols}: (; ); . (\texttt{dot}); , (\texttt{comma})
                    \end{itemize}
                \end{frame}

                \begin{frame}
                    \frametitle{Syntax - Non-logical symbols}
                    \begin{itemize}
                        \item Represents \textit{predicates} (or \textit{relations}), \textit{functions} and \textit{constants}
                        \item $ \forall n \in \mathbb{Z^*} $ we have a set of \textit{n-ary} \textbf{predicate symbols}
                        \[
                            P^n_0, P^n_1, \dots \quad\quad \text{(Uppercase letters)}
                        \] 
                        \item $ \forall n \in \mathbb{Z^*} $ there exist \underline{infinite} \textit{n-ary} \textbf{function symbols}
                        \[
                            f^n_0, f^n_1, \dots \quad\quad \text{(Lowercase letters)}
                        \]
                    \end{itemize}
                \end{frame}

                \begin{frame}
                    \frametitle{Syntax - Formation rules (1)}
                    \begin{definition}[Terms formation]
                        The set \texttt{TERM} of \textit{terms} can be inductively defined by the following rules:
                        \begin{enumerate}
                            \item If $ x $ is a variable, then $ x \in \texttt{TERM}$
                            \item Any expression $ f(t_1, \dots, t_n) $, with $ t_1, \dots, t_n \in \texttt{TERM} $, is a term.\\
                            Since that, the following statement holds
                            \[
                                f(t_1, \dots, t_n) \in \texttt{TERM}
                            \]
                        \end{enumerate}
                    \end{definition}
                \end{frame}

                \begin{frame}
                    \frametitle{Syntax - Formation rules (2)}
                    \begin{definition}[Formulas formation]
                        The set \texttt{FORM} of \textit{formulas} can be inductively defined by the following rules:
                        \begin{enumerate}
                            \item If $ P \in \texttt{PRED}\footnote{The set of \textit{predicate symbols}} $ and $ t_1, \dots, t_n \in \texttt{TERM} $, than $ P(t_1, \dots, t_n) \in \texttt{FORM} $
                            \item If $ t_1, t_2 \in \texttt{TERM} $, than $ t_1 = t_2 \in \texttt{FORM} $
                            \item If $ \phi \in \texttt{FORM} $, than $ \neg \phi \in \texttt{FORM} $
                            \item If $ \phi, \psi \in \texttt{FORM} $, than $ \phi\; \square\; \psi \in \texttt{FORM} $ (with $ \square \in \{ \wedge, \vee, \Rightarrow \} $)
                            \item If $ \phi \in \texttt{FORM} $ and $ x $ is a variable, than $ Qx.\phi \in \texttt{FORM} $ (with $ Q \in \{ \forall, \exists \} $)
                        \end{enumerate}
                    \end{definition}
                \end{frame}

                \begin{frame}
                    \frametitle{Syntax - Variables (1)}
                    \begin{definition}[Free and Bound variables]
                        The \textit{free} and \textit{bound} variable occurrences in a formula are defined inductively by the following rules:
                        \begin{enumerate}
                            \item If $ \phi $ is \textit{atomic}, than any variable $ x \in Var(\phi) $ is \textit{free}
                            \item $ x $ is \textit{free/bound} in $ \neg \phi $ iff $ x $ is \textit{free/bound} in $ \phi $
                            \item $ x $ is \textit{free/bound} in $ \phi\; \square\; \psi $ iff $ x $ is \textit{free/bound} in either $ \phi $ or $ \psi $ (with $ \square \in \{ \wedge, \vee, \Rightarrow \} $)
                            \item $ x $ is \textit{free} in $ Qy.\phi $ iff $ x $ is \textit{free} in $ \phi $ and $ y \neq x $
                            \item $ x $ is \textit{bound} in $ Qy.\phi $ iff $ x $ is \textit{bound} in $ \phi $
                        \end{enumerate}
                    \end{definition}
                \end{frame}

                \begin{frame}
                    \frametitle{Syntax - Variables (2)}
                    \begin{itemize}
                        \item More easily, a variable $ x $ is \textit{bounded} if it occurs in a quantification, $ x $ is \textit{free} otherwise
                        \item A variable can be both \textit{free} and \textit{bounded} in the same formula, e.g.
                        \[
                            P(x, y) \Rightarrow \exists x.Q(x)    
                        \]
                        \begin{enumerate}
                            \item In the \textbf{LHS} $ x $ is \textit{free}
                            \item In the \textbf{RHS} $ x $ is \textit{bounded}
                            \item Even so, the formula is still \textit{well-formed}
                        \end{enumerate}
                        \item A formula with no \textit{free} variables is called a \textbf{sentence}
                    \end{itemize}
                \end{frame}

            \subsubsection{Semantics}
                \begin{frame}
                    \frametitle{Semantics - Introduction}
                    \begin{itemize}
                        \item Interpretation defines the \textbf{domain} $ D $, a non-empty set of "\textit{objects}"
                        \item Interpretation \textit{maps} functions, predicates and constants symbols into elements of $ D $
                    \end{itemize}
                \end{frame}

                \begin{frame}
                    \frametitle{Semantics - Structure}
                    \begin{definition}[Structure]
                         A \textit{structure} is formed by a \textit{domain} $ D $ and an \textit{interpretation function} $ I $ s.t.
                         \begin{enumerate}
                            \item Each function symbol $ f $ of arity $ n $ is assigned to a function s.t.
                            \[
                                I(f): D^n \to D 
                            \]
                            \item Each predicate symbol $ f $ of arity $ n $ is assigned to a function s.t. 
                            \[
                                I(P): D^n \to \{\texttt{true}, \texttt{false}\}
                            \]
                         \end{enumerate}
                    \end{definition}
                \end{frame}
\end{document}
