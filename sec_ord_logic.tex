\documentclass{beamer}

\usepackage{amssymb}
\usepackage{amsthm}
\usepackage{bussproofs}
\usepackage{cancel}
\usepackage{stmaryrd}
\usepackage{xcolor}

\newenvironment{proofidea}{
    \renewcommand{\proofname}{Proof idea.}\proof
}{\endproof}

\newcommand{\bigdoublewedge}{
  \mathop{
    \mathchoice{\bigwedge\mkern-15mu\bigwedge}
               {\bigwedge\mkern-12.5mu\bigwedge}
               {\bigwedge\mkern-12.5mu\bigwedge}
               {\bigwedge\mkern-11mu\bigwedge}
    }
}

\newtheorem{property}{Property}
\newtheorem{proposition}{Proposition}

\definecolor{title_blue}{RGB}{51, 51, 178}

\setbeamertemplate{navigation symbols}{}

\usetheme{Boadilla}

\title{Second-Order Logic}
\subtitle{Final seminar for ``Logic in Computer Science"}
\author{Mattia Zorzan}
\institute{University of Verona}
\date{Last rev. \today}

\begin{document}
    \begin{frame}
        \titlepage
    \end{frame}

    \begin{frame}
        \frametitle{Outline}
        \tableofcontents
    \end{frame}

    \section{Introduction}
        \begin{frame}
            \frametitle{Introduction}
            \begin{itemize}
                \item First-order logic allows ``iteration" over the \textit{elements} of a structure
                \item Happens thanks to \textbf{quantifiers}: $ \forall $, $ \exists $
                    \begin{itemize}
                        \item $ \forall x. \phi(x) \rightarrow $ ``For each $ x $, $ x $ satisfies the formula $ \phi $"
                        \item $ \exists x. \phi(x) \rightarrow $ ``There exists $ x $ s.t. the formula $ \phi $ is satisfied"
                    \end{itemize}
                \item Limiting since we may only need to range over \textit{subsets} or ``\textit{combinations}" (e.g. \textit{Cartesian product})
            \end{itemize}
        \end{frame}

        \subsection{A recall on first-order logic}
            \begin{frame}
                \frametitle{A brief recall}
                \begin{itemize}
                    \item Second-order logic ``extends" first-order logic
                    \item Since that, let's recall the basics of first-order logic
                    \item Two key parts:
                        \begin{itemize}
                            \item \textit{Syntax}: Which sequences constitute \textbf{well-formed} expressions
                            \item \textit{Semantic}: The \textbf{meaning} behind this expressions
                        \end{itemize}
                \end{itemize}
            \end{frame}

            \subsubsection{Syntax}
                \begin{frame}
                    \vspace*{1cm}
                    \begin{center}
                        \begin{Huge}
                            \textcolor{title_blue}{Syntax}
                        \end{Huge}
                    \end{center}
                \end{frame}

                \begin{frame}
                    \frametitle{Introduction}
                    \begin{itemize}
                        \item Two base types:
                            \begin{itemize}
                                \item \textbf{Terms}: Represents \textit{objects}
                                \item \textbf{Formulas}: Represents \textit{predicates}
                            \end{itemize}
                        \item Both formed by \textit{symbol} concatenation
                        \item All symbols together form the \textbf{alphabet} of the language
                        \item Can divide symbols in two categories
                            \begin{itemize}
                                \item \textit{Logical} symbols
                                \item \textit{Non-logical} symbols
                            \end{itemize}
                    \end{itemize}
                \end{frame}

                \begin{frame}
                    \frametitle{Logical symbols}
                    \begin{itemize}
                        \item \underline{Infinite} set of \textbf{variables}: $ x, y, z, \dots, x_0, x_1, \dots $ (Lowercase letters)
                        \item \textbf{Connectives}: $ \wedge, \vee, \Rightarrow, \neg $
                        \item \textbf{Quantifiers}: $ \forall, \exists $
                        \item \textbf{Equality} (or \textit{Identity}): $ = $
                        \item \textbf{Auxiliary symbols}: (; ); . (\texttt{dot}); , (\texttt{comma})
                    \end{itemize}
                \end{frame}

                \begin{frame}
                    \frametitle{Non-logical symbols}
                    \begin{itemize}
                        \item Represents \textit{predicates} (or \textit{relations}), \textit{functions} and \textit{constants}
                        \item $ \forall n \in \mathbb{Z^*} $ we have a set of \textit{n-ary} \textbf{predicate symbols}
                        \[
                            P^n_0, P^n_1, \dots \quad\quad \text{(Uppercase letters)}
                        \] 
                        \item $ \forall n \in \mathbb{Z^+} $ there exist \underline{infinite} \textit{n-ary} \textbf{function symbols}
                        \[
                            f^n_0, f^n_1, \dots \quad\quad \text{(Lowercase letters)}
                        \]
                        \item \textbf{Constants} can be seen as \textit{0-ary} functions: $ c_0, c_1, \dots $
                    \end{itemize}
                \end{frame}

                \begin{frame}
                    \frametitle{Formation rules (1)}
                    \begin{definition}[Terms formation]
                        The set \texttt{TERM} of \textit{terms} can be inductively defined by the following rules:
                        \begin{enumerate}
                            \item If $ x $ is a variable, then $ x \in \texttt{TERM}$
                            \item Any expression $ f(t_1, \dots, t_n) $, with $ t_1, \dots, t_n \in \texttt{TERM} $, is a term.\\
                            Since that, the following statement holds
                            \[
                                f(t_1, \dots, t_n) \in \texttt{TERM}
                            \]
                        \end{enumerate}
                    \end{definition}
                \end{frame}

                \begin{frame}
                    \frametitle{Formation rules (2)}
                    \begin{definition}[Formulas formation]
                        The set \texttt{FORM} of \textit{formulas} can be inductively defined by the following rules:
                        \begin{enumerate}
                            \item If $ P $ is a \textit{predicate} and $ t_1, \dots, t_n \in \texttt{TERM} $, than $ P(t_1, \dots, t_n) \in \texttt{FORM} $
                            \item If $ t_1, t_2 \in \texttt{TERM} $, than $ (t_1 = t_2) \in \texttt{FORM} $
                            \item If $ \phi \in \texttt{FORM} $, than $ \neg \phi \in \texttt{FORM} $
                            \item If $ \phi, \psi \in \texttt{FORM} $, than $ (\phi\; \square\; \psi) \in \texttt{FORM} $ (with $ \square \in \{ \wedge, \vee, \Rightarrow \} $)
                            \item If $ \phi \in \texttt{FORM} $ and $ x $ is a variable, than $ Qx.\phi \in \texttt{FORM} $ (with $ Q \in \{ \forall, \exists \} $)
                        \end{enumerate}
                    \end{definition}
                \end{frame}

                \begin{frame}
                    \frametitle{Variables (1)}
                    \begin{definition}[Free and Bound variables]
                        The \textit{free} and \textit{bound} variable occurrences in a formula are defined inductively by the following rules:
                        \begin{enumerate}
                            \item If $ \phi $ is \textit{atomic}, than any variable $ x \in Var(\phi) $ is \textit{free}
                            \item $ x $ is \textit{free/bound} in $ \neg \phi $ iff $ x $ is \textit{free/bound} in $ \phi $
                            \item $ x $ is \textit{free/bound} in $ (\phi\; \square\; \psi) $ iff $ x $ is \textit{free/bound} in either $ \phi $ or $ \psi $ (with $ \square \in \{ \wedge, \vee, \Rightarrow \} $)
                            \item $ x $ is \textit{free} in $ Qy.\phi $ iff $ x $ is \textit{free} in $ \phi $ and $ y \neq x $
                            \item $ x $ is \textit{bound} in $ Qy.\phi $ iff $ x $ is \textit{bound} in $ \phi $
                        \end{enumerate}
                    \end{definition}
                \end{frame}

                \begin{frame}
                    \frametitle{Variables (2)}
                    \begin{itemize}
                        \item More easily, a variable $ x $ is \textit{bounded} if it occurs in a quantification, $ x $ is \textit{free} otherwise
                        \item A variable can be both \textit{free} and \textit{bounded} in the same formula, e.g.
                        \[
                            P(x, y) \Rightarrow \exists x.Q(x)    
                        \]
                        \begin{enumerate}
                            \item In the \textbf{LHS} $ x $ is \textit{free}
                            \item In the \textbf{RHS} $ x $ is \textit{bounded}
                            \item Even so, the formula is still \textit{well-formed}
                        \end{enumerate}
                        \item A formula with no \textit{free} variables is called a \textbf{sentence}
                    \end{itemize}
                \end{frame}

            \subsubsection{Semantic}
                \begin{frame}
                    \vspace*{1cm}
                    \begin{center}
                        \begin{Huge}
                            \textcolor{title_blue}{Semantic}
                        \end{Huge}
                    \end{center}
                \end{frame}

                \begin{frame}
                    \frametitle{Structure and Interpretation}
                    \begin{definition}[Structure]
                        A \textit{structure} is formed by a \textit{domain} $ D $, $\mathbb{P} = \{ P_1, \dots, P_n \} $ predicates on $ D $, $ \mathbb{F} = \{ f_1, \dots, f_n \} $ \textbf{total} functions on $ D $ and a set $ \mathbb{C} \subseteq D $ of constants
                    \end{definition}
                    \begin{definition}[Interpretation]
                        Given a \textit{structure} $ \mathfrak{D} $ and a \textit{map} $ (\cdot)^{\mathfrak{D}} $ s.t.
                        \begin{itemize}
                            \item for all $ c $ in my language, $ (c)^\mathfrak{D} = c^\mathfrak{D} \in \mathbb{C} $
                            \item for all \textit{k-ary} function $ f $ in my language, $ (f)^\mathfrak{D} = f^\mathfrak{D}: D^k \to D \in \mathbb{F} $
                            \item for all \textit{n-ary} predicate $ P $ in my language, $ (P)^\mathfrak{D} = P^\mathfrak{D} \subseteq D^n \in \mathbb{P} $
                        \end{itemize}
                        we call $ \langle \mathfrak{D}, (\cdot)^{\mathfrak{D}} \rangle $ an \textit{interpretation}.
                    \end{definition}

                \end{frame}

                \begin{frame}
                    \frametitle{Evaluation (1)}
                    \begin{itemize}
                        \item Given an \textit{interpretation} and an \textbf{assignment} $ \overline{a} $, it is possible to evaluate a formula
                        \item The \textbf{evaluation} process \textit{maps} the whole formula to a \textbf{truth value}
                        \item The \textit{assignment} $ \overline{a} $ associates each \textit{free variable} with a \textit{truth value}
                        \item If the formula is a \textit{sentence}, $ \overline{a} $ does not affect the \textit{truth value} of the formula
                        \item Next slides shows the evaluation steps   
                    \end{itemize}
                \end{frame}

                \begin{frame}
                    \frametitle{Evaluation (2)}
                    \begin{enumerate}
                        \item Extend $ \overline{a} $ to all terms of the language with the following rules:
                            \begin{itemize}
                                \item Each variable $ x $ evaluates to $ \overline{a}(x) $
                                \item Given $ \{ t_1, \dots, t_n \} \in \texttt{TERM} $ evaluated to $ \{ d_1, \dots, d_n \} $, a function $ f(t_1, \dots, t_n) $ evaluates to $ (f)^{\mathfrak{D}}\footnote{Supposing we're evaluating $ f $ in a structure $ \mathfrak{D} $}(d_1, \dots, d_n) $ 
                            \end{itemize}
                        \item Assign each formula to a \textit{truth value} with the following (inductive) rules:
                            \begin{itemize}
                                \item (\textit{Continues in next slides})
                            \end{itemize} 
                    \end{enumerate}
                \end{frame}

                \begin{frame}
                    \frametitle{Evaluation (3)}
                    \begin{itemize}
                        \item An \textit{atomic formula} $ P(t_1, \dots, t_n) $ is associated with a \textit{truth value}, depending on the truth of the following:
                            \[
                                \langle v_1, \dots, v_n \rangle \in (P)^{\mathfrak{D}}   
                            \]
                            where $ v_1, \dots, v_n $ represents the evaluation of the predicate terms
                        \item An \textit{atomic formula} $ (t_1 = t_2) $ evaluates to a \textit{truth value} depending if $ v_1 = v_2 $ in $ D $, where $ v_1, v_2 $ represents the evaluation of the terms
                    \end{itemize}
                \end{frame}

                \begin{frame}
                    \frametitle{Evaluation (4)}
                    \begin{itemize}
                        \item A formula containing \textit{logical connectives} (e.g. $ \phi\; \square\; \psi\footnote{with $ \square \in \{ \wedge, \vee, \Rightarrow \} $}, \neg \phi $) is evaluated according to the \textit{truth table} of the connective
                        \item A formula $ \exists x.\phi $ is evaluated \texttt{true} iff exists an assignment $ \overline{a}' $ s.t. it differs from $ \overline{a} $ only for the assignment of $ x $ and $ \phi $ is evaluated \texttt{true} via the $ \overline{a}' $ assignment, \texttt{false} otherwise
                        \item A formula $ \forall x. \phi $ is evaluated \texttt{true} iff exists an assignment $ \overline{a}' $ s.t. it differs from $ \overline{a} $ only for the assignment of $ x $ and $ \phi $ is evaluated \texttt{true} for all values in $ \overline{a}' $, \texttt{false} otherwise
                    \end{itemize}
                \end{frame}

                \begin{frame}
                    \frametitle{Evaluation (5)}
                    \begin{itemize}
                        \item Given a \textit{structure} $ \mathfrak{D} $, evaluation can be seen as a \textit{map}
                        \[
                            \rho_\mathfrak{D}: Var \to D
                        \]
                        \item We can use the $ \llbracket \cdot \rrbracket $ notation to express the evaluation of a \textit{term}
                        \begin{definition}
                            Given a structure $ \mathfrak{D} $ and $ \llbracket \cdot \rrbracket_{\rho_\mathfrak{D}}: \texttt{TERM} \to D $, we can define
                            \begin{enumerate}
                                \item $ \llbracket c \rrbracket_{\rho_\mathfrak{D}} = c^\mathfrak{D}$
                                \item $ \llbracket x \rrbracket_{\rho_\mathfrak{D}} = \rho_\mathfrak{D}(x) $
                                \item $ \llbracket f(t_1, \dots, t_n) \rrbracket_{\rho_\mathfrak{D}} = f^\mathfrak{D}(\llbracket t_1 \rrbracket_{\rho_\mathfrak{D}}, \dots, \llbracket t_n \rrbracket_{\rho_\mathfrak{D}}) $
                            \end{enumerate}
                        \end{definition}
                    \end{itemize}
                \end{frame}

                \begin{frame}
                    \frametitle{Satisfiability}
                    \begin{definition}[Satisfiability relation]
                        Given a \textit{structure} $ \mathfrak{D} $ we can recursively define the \textit{satisfiability relation} $ \vDash $ as:
                        \begin{enumerate}
                            \item $ \rho_\mathfrak{D}\; \cancel{\vDash}\; \bot $
                            \item $ \rho_\mathfrak{D} \vDash P(t_1, \dots, t_n) \Leftrightarrow \langle \llbracket t_1 \rrbracket_{\rho_\mathfrak{D}}, \dots, \llbracket t_n \rrbracket_{\rho_\mathfrak{D}} \rangle \in \mathbb{P} $
                            \item $ \rho_\mathfrak{D} \vDash (t_1 = t_2) \Leftrightarrow\llbracket t_1 \rrbracket_{\rho_\mathfrak{D}} = \llbracket t_2 \rrbracket_{\rho_\mathfrak{D}} $
                            \item $ \rho_\mathfrak{D} \vDash (\phi \Rightarrow \psi) \Leftrightarrow \rho_\mathfrak{D}\; \cancel{\vDash}\; \phi \text{ o } \rho_\mathfrak{D} \vDash \psi $
                            \item $ \rho_\mathfrak{D} \vDash (\phi \wedge \psi) \Leftrightarrow \rho_\mathfrak{D} \vDash \phi \text{ e } \rho_\mathfrak{D} \vDash \psi $
                            \item $ \rho_\mathfrak{D} \vDash (\phi \vee \psi) \Leftrightarrow \rho_\mathfrak{D} \vDash \phi \text{ o } \rho_\mathfrak{D} \vDash \psi $
                            \item $ \rho_\mathfrak{D} \vDash \forall x.\, \phi(x) \Leftrightarrow \forall a \in D: \rho_\mathfrak{D}[x/a] \vDash \phi $
                            \item $ \rho_\mathfrak{D} \vDash \exists x.\, \phi(x) \Leftrightarrow \exists a \in D: \rho_\mathfrak{D}[x/a] \vDash \phi $
                        \end{enumerate}
                    \end{definition}
                \end{frame}

                \begin{frame}
                    \frametitle{Natural Deduction}
                    \begin{itemize}
                        \item Just as reference, we introduce the \textit{quantification rules} for the First-Order natural deduction system
                    \end{itemize}
                    \begin{columns}[c]
                        \begin{column}{.5\textwidth}
                            \begin{prooftree}
                                \AxiomC{$ \phi(x) $}
                                \RightLabel{$ \forall I $}
                                \UnaryInfC{$ \forall y. \phi(y) $}
                            \end{prooftree}
                            \begin{prooftree}
                                \AxiomC{$ \phi(y) $}
                                \RightLabel{$ \exists I $}
                                \UnaryInfC{$ \exists x. \phi(x) $}
                            \end{prooftree}
                        \end{column}
                        \begin{column}{.5\textwidth}
                            \begin{prooftree}
                                \AxiomC{$ \forall x. \phi(x) $}
                                \RightLabel{$ \forall E $}
                                \UnaryInfC{$ \phi(y) $}
                            \end{prooftree}
                            \begin{prooftree}
                                \AxiomC{$ \exists y. \phi(y) $}
                                \AxiomC{[$ \phi(x) $]}
                                \noLine
                                \UnaryInfC{$ \Pi $}
                                \noLine
                                \UnaryInfC{$ \psi $}
                                \RightLabel{$ \exists E $}
                                \BinaryInfC{$ \psi $}
                            \end{prooftree}
                        \end{column}
                    \end{columns}
                \end{frame}

        \section{Second-Order Logic}
            \begin{frame}
                \begin{center}
                    \vspace*{1cm}
                    \begin{Huge}
                        \textcolor{title_blue}{Second-Order Logic}
                    \end{Huge}
                \end{center}
            \end{frame}

            \begin{frame}
                \frametitle{Introduction}
                \begin{itemize}
                    \item As said before, Second-Order Logic \textit{extends} First-Order Logic (respectively SOL and FOL from now on)
                    \item The \textit{terms} and \textit{formulas} are pretty much the same of FOL
                    \item To represent this ``\textit{extension}" we need a brief redefinition of what we've seen in the previous section
                \end{itemize}
            \end{frame}

            \subsection{Syntax}
                \begin{frame}
                    \vspace*{1cm}
                    \begin{center}
                        \begin{Huge}
                            \textcolor{title_blue}{Syntax}
                        \end{Huge}
                    \end{center}
                \end{frame}

                \begin{frame}
                    \frametitle{Alphabet}
                    \begin{itemize}
                        \item Consists of the following symbols:
                              \begin{itemize}
                                \item \textbf{Individual variables}: $ x_0, x_1, \dots $
                                \item \textbf{Individual constants}: $ c_0, c_1, \dots $
                                \item \textbf{Predicate variables}: $ X_0^n, X_1^n, \dots $
                                \item \textbf{Predicate constants}: $ \bot, P_0^n, P_1^n, \dots $
                                \item \textbf{Connectives}: $ \wedge, \vee, \Rightarrow, \neg $
                                \item \textbf{Quantifiers}: $ \forall, \exists $
                                \item \textbf{Auxiliary symbols}: (; ); . (\texttt{dot}); , (\texttt{comma})
                              \end{itemize}
                    \end{itemize}
                \end{frame}

                \begin{frame}
                    \frametitle{Formulas}
                    \begin{itemize}
                        \item The set $ \texttt{FORM}_2 $ of Second-Order \textit{formulas} is inductively defined as follows
                              \begin{itemize}
                                \item $ X^0_i, P^0_i, \bot \in \texttt{FORM}_2 $
                                \item $ \forall n \in \mathbb{Z}^+. X^n(t_1, \dots, t_n) \in \texttt{FORM}_2 $
                                \item $ \forall n \in \mathbb{Z}^+. P^n(t_1, \dots, t_n) \in \texttt{FORM}_2 $
                              \end{itemize}
                        \item $ \texttt{FORM}_2 $ is \textit{closed} under \textbf{connectives} and \textbf{quantifiers}
                    \end{itemize}
                \end{frame}

            \subsection{Semantic}
                \begin{frame}
                    \vspace*{1cm}
                    \begin{center}
                        \begin{Huge}
                            \textcolor{title_blue}{Semantic}
                        \end{Huge}
                    \end{center}
                \end{frame}

                \begin{frame}
                    \frametitle{Structure}
                    \begin{definition}[Second-Order Structure]
                        A \textit{Second-Order structure} is formed by a \textit{domain} $ D $, a set $ D^* = \{ D_n\; \vert\; n \in \mathbb{N} \}e$ with $ D_n \subseteq \mathcal{P}(D^n) $, a set $ \mathbb{R} = \{ R_1^n, \dots, R_k^n \} $ of \textit{predicates} s.t. $ R_i^n \in D_n $ and a set of \textit{constants} $ \mathbb{C} \subseteq D $
                    \end{definition}
                    \begin{itemize}
                        \item If $ D_n $ contains \textbf{all} n-ary predicates ($ D_n = \mathcal{P}(D^n) $) we call the structure \textit{full}
                        \item Even if the \textit{elements} of a Second-Order structure are slightly different from the elements of a First-Order structure, we can use the same rules for \textbf{interpretation} and \textbf{evaluation}
                    \end{itemize}
                \end{frame}
                
                \begin{frame}
                    \frametitle{Satisfiability}
                    \begin{definition}[Second-Order Satisfiability relation]
                        Given a \textit{Second-Order structure} $ \mathfrak{D}_2 $ and a \textit{language} $ \mathcal{L} $ that defines a name $ \overline{S} $ for all $ S \in D $, we can define the \textit{satisfiability relation} $ \vDash_2 $ as:
                        \begin{enumerate}
                            \item $ \rho_{\mathfrak{D}_2}\; \cancel{\vDash_2}\; \bot $
                            \item $ \rho_{\mathfrak{D}_2} \vDash_2 \overline{S}^n(\overline{s}_1, \dots, \overline{s}_n) \Leftrightarrow \langle s_1, \dots, s_n \rangle \in S^n $\footnote{By using this notation, we can handle all types of \textit{predicates} with one rule}
                            \item All connectives follow the same rules of First-Order Logic
                            \item Quantification over a \textit{variable} follow the same rules of First-Order Logic
                            \item $ \rho_{\mathfrak{D}_2} \vDash_2 \forall X^n_i. \phi(X^n_i) \Leftrightarrow \forall S^n \in D_n: \rho_{\mathfrak{D}_2} \vDash_2 \phi(S^n) $
                            \item $ \rho_{\mathfrak{D}_2} \vDash_2 \exists X^n_i. \phi(X^n_i) \Leftrightarrow \exists S^n \in D_n: \rho_{\mathfrak{D}_2} \vDash_2 \phi(S^n) $
                        \end{enumerate}
                    \end{definition}
                \end{frame}
                
                \begin{frame}
                    \frametitle{Natural Deduction}
                    \begin{itemize}
                        \item We need to add a set of \textit{rules} that allows to validly \textbf{derive} the new ``extended" \textit{quantifications}
                    \end{itemize}
                    \begin{columns}[c]
                        \begin{column}{0.5\textwidth}
                            \begin{prooftree}
                                \AxiomC{$ \phi $}
                                \RightLabel{$ \forall^2I $}
                                \UnaryInfC{$ \forall X^n.\phi $}
                            \end{prooftree}
                            \begin{prooftree}
                                \AxiomC{$ \phi^* $}
                                \RightLabel{$ \exists^2I $}
                                \UnaryInfC{$ \exists X^n.\phi $}
                            \end{prooftree}
                        \end{column}
                        \hfill
                        \begin{column}{0.5\textwidth}
                            \begin{prooftree}
                                \AxiomC{$ \forall X^n.\phi $}
                                \RightLabel{$ \forall^2E $}
                                \UnaryInfC{$ \phi^* $}
                            \end{prooftree}
                            \begin{prooftree}
                                \AxiomC{$ \exists X^n. \phi $}
                                \AxiomC{$ [\phi] $}
                                \noLine
                                \UnaryInfC{$ \Pi $}
                                \noLine
                                \UnaryInfC{$ \psi $}
                                \RightLabel{$ \exists^2E $}
                                \BinaryInfC{$ \psi $}
                            \end{prooftree}
                        \end{column}
                    \end{columns}
                    \vspace{.5cm}
                    \begin{itemize}
                        \item $ \phi^* $ is $ \phi[X^n(t_1, \dots, t_n) / \psi(t_1, \dots, t_n)] $, where $ \psi $ is a generic formula
                        \item No $ t_i $ becomes \textit{bounded} during the substitution above, so $ \psi $ cannot \textit{quantify} any term $ t_1, \dots, t_n $
                    \end{itemize}
                \end{frame}

                \begin{frame}
                    \frametitle{Derivability}
                    \begin{itemize}
                        \item We need to introduce another relation that states when a \textit{formula} is \textit{derivable}
                    \end{itemize}
                    \begin{definition}[Second-Order Derivability relation]
                        Given a \textit{deduction system} and a set of \textit{formulas} $ \Gamma $, we can say $ \Gamma \vdash_2 \phi $ (read as ``\textit{Gamma derives phi}") iff starting from $ \Gamma $ we can derive $ \phi $ using the \textit{deduction system}.\\
                        Formally, if exists a derivation
                        \begin{prooftree}
                            \AxiomC{$ \Pi $}
                            \UnaryInfC{$ \phi $}
                        \end{prooftree}
                        s.t. $ Hp[\Pi] \subseteq \Gamma $
                    \end{definition}
                    \begin{itemize}
                        \item Same as in \textit{First-Order Logic}
                    \end{itemize}
                \end{frame}

                \begin{frame}
                    \frametitle{Comprehension schema}
                    \begin{itemize}
                        \item States that ``Any \underline{definable} subclass of a set is a set", formally:
                    \end{itemize}
                    \begin{definition}[Axiom schema of comprehension]
                        Given a \textit{formula} $ \psi $ with $ FV[\psi] \in \{ x, t_1, \dots, t_n \} $,  and a set $ A $, the following holds
                        \[
                            \forall t_1, \dots, t_n. \forall A. \exists B. \forall x. (x \in B \Leftrightarrow (x \in A \wedge \psi(x, t_1, \dots, t_n, A)))    
                        \]
                    \end{definition}
                    \begin{itemize}
                        \item Since the schema holds for all $ x $ and for all $ A $, for the generality of $ \psi $, it is always possible to define a set from another set
                        \item $ \exists^2I $ gives us this schema, since we can substitute $ X^n $ with $ \psi $, obtaining the ``\textit{filtered}" subset
                    \end{itemize}
                \end{frame}

            \subsection{Flattening SOL}
                \begin{frame}
                    \frametitle{Flattening SOL (1)}
                    \begin{itemize}
                        \item Given the \textit{First-Order predicates} $ Ap_0, Ap_1, \dots $ s.t. $ Ap_n $ is $ (n + 1) $-ary, since it comprehends the \textit{symbol} of the predicate and it's \textit{arguments}
                        \item $ Ap_n(X, t_1, \dots, t_n) $ can be seen as $ X^n(t_1, \dots, t_n) $, so $ Ap_0 $ is the First-Order version of $ X^0 $
                        \item We also use some \textit{unary predicates}
                              \begin{itemize}
                                \item $ V \rightarrow $ ``is an element"
                                \item $ U_0 \rightarrow $ ``is a 0-ary predicate"
                                \item $ U_1 \rightarrow $ ``is a 1-ary predicate"
                                \item \textit{and so on...}
                              \end{itemize}
                    \end{itemize}
                \end{frame}

                \begin{frame}
                    \frametitle{Flattening SOL (2)}
                    The following axioms embody the \textit{characteristic properties} of SOL
                    \vspace*{.5cm}
                    \begin{enumerate}
                        \item $ \forall x, y, z.(U_i(x) \wedge U_j(y) \wedge V(z) \Rightarrow x \neq y \wedge y \neq z \wedge z \neq x) $ for all $ i \neq j $
                        \item $ \forall X, y_1, \dots, y_n.(Ap_n(X, y_1, \dots, y_n) \Rightarrow U_n(X) \wedge \bigdoublewedge_i V(y_i)) $ for $ n \geq 1 $
                        \item $ U_0(C_0, V(C_{2^{i + 1}})) $ for $ i \geq 0 $ and $ U_n(C_{3^i5^n}) $ for $ i, n \geq 0 $
                        \item $ \begin{aligned}
                            \forall &z_1, \dots, z_m. \exists X.(\\
                            &\quad U_n(X) \wedge \forall y_1, \dots, y_n. \Big (\bigdoublewedge V(y_i) \Rightarrow (\phi^* \Leftrightarrow Ap_n(X, y_1, \dots, y_n))\Big )\\
                            ) \end{aligned} $ \\[-1\baselineskip]\quad where $ X \not\in FV[\phi^*] $, $ FV[\phi] \subseteq \{ z_1, \dots, z_m, y_1, \dots, y_n \} $
                        \item $ \neg Ap_0(C_0) $
                    \end{enumerate}
                \end{frame}

                \begin{frame}
                    \frametitle{Flattening SOL (3)}
                    \begin{enumerate}
                        \item The $ i $-ary and $ j $-ary predicates are pairwise disjoint and disjoint from the \textit{element}
                        \item If $ \langle X, y_1, \dots, y_n \rangle \in Ap_n $ then $ X $ is a \textit{predicate} and $ y_1, \dots, y_n $ are it's \textit{elements}
                        \item We can have both \textit{element} and \textit{predicate} constants
                        \item First-Order equivalent of the \textit{comprehension schema}
                        \item The 0-ary predicate for ``\texttt{false}", equivalent to $ \bot $
                    \end{enumerate}
                \end{frame}

                \begin{frame}
                    \frametitle{Flattening SOL (4)}
                    \begin{itemize}
                        \item SOL can be translated preserving \textbf{derivability}
                        \item We can assign symbols to the ones in SOL alphabet, in order to convert strings inductively
                        \begin{itemize}
                            \item $ (x_i)^* \leftarrow x_{2^{i + 1}} $
                            \item $ (c_i)^* \leftarrow c_{2^{i + 1}} $
                            \item $ (X^n_i)^* \leftarrow x_{3^i5^n} $
                            \item $ (P^n_i)^* \leftarrow c_{3^i5^n} $
                            \item $ (X^0_i)^* \leftarrow Ap_0(x_{3^i}) $
                            \item $ (P^0_i)^* \leftarrow Ap_0(c_{3^i}) $
                            \item $ (\bot)^* \leftarrow Ap_0(c_0) $
                        \end{itemize}
                        with $ i,n \geq 0 $
                    \end{itemize}
                \end{frame}

                \begin{frame}
                    \frametitle{Flattening SOL (5)}
                    \begin{itemize}
                        \item Doing this, we can ``\textit{translate}" Second-Order formulas as follows:
                              \begin{itemize}
                                \item $ (\phi\, \square\, \psi)^* \leftarrow \phi^*\, \square\, \psi^* $
                                \item $ (\neg \phi)^* \leftarrow \neg \phi^* $
                                \item $ (\forall x_i. \phi(x_i))^* \leftarrow \forall x_i^*. (V(x_i^*) \Rightarrow \phi^*(x_i^*)) $
                                \item $ (\exists x_i. \phi(x_i))^* \leftarrow \exists x_i^*. (V(x_i^*) \wedge \phi^*(x_i^*)) $
                                \item $ (\forall X^n_i. \phi(X^n_i))^* \leftarrow \forall (X^n_i)^*. (U_n((X^n_i)^*) \Rightarrow \phi^*((X^n_i)^*)) $
                                \item $ (\exists X^n_i. \phi(X^n_i))^* \leftarrow \exists (X^n_i)^*. (U_n((X^n_i)^*) \wedge \phi^*((X^n_i)^*)) $
                              \end{itemize}
                        \item This further strengthens the result of \textit{derivability}, since we can state that $ \vdash\, \Rightarrow\, \vdash_2 $
                    \end{itemize}
                \end{frame}

            \subsection{Soundness and Completeness}

                \begin{frame}
                    \frametitle{Soundness (1)}
                    \begin{itemize}
                        \item Given the \textit{derivability relation} $ \vdash_2 $, it is easy to prove the \textbf{Soundness} result
                        \item Just recall that by \textit{flattening} we just proved that we can translate each \textit{Second-Order Formula} into a \textit{First-Order formula}
                        \item Since the result of derivability \textit{holds} for SOL the same way it holds for FOL, we can state the following
                    \end{itemize}
                    \begin{corollary}
                        Given a formula $ \phi $, $ \vdash \phi \Rightarrow\; \vdash_2 \phi $ since we never added a derivation
                        \begin{prooftree}
                            \AxiomC{$ \Pi $}
                            \UnaryInfC{$ \bot $}
                        \end{prooftree}
                        in extending FOL \textit{Natural Deduction}, so $ \cancel{\vdash_2}\, \bot $
                    \end{corollary}
                \end{frame}

                \begin{frame}
                    \frametitle{Soundness (2)}
                    \begin{theorem}[Soundness]
                        $ \Gamma \vdash_2 \phi \Rightarrow \Gamma \vDash_2 \phi $
                    \end{theorem}
                    \begin{proofidea}
                        Proof by induction on derivations.\\
                        From the \textit{corollary} seen in tre previous slide $ \cancel{\vdash_2}\, \bot $, so $ \Gamma \vdash_2 \phi \Rightarrow \phi \neq \bot $.\\
                        If $ \Gamma $ is formed by any \textit{Second-Order formula}, we can say that $ \Gamma \vDash_2 \phi $ since we can find a derivation
                        \begin{prooftree}
                            \AxiomC{$ \Pi $}
                            \UnaryInfC{$ \phi $}
                        \end{prooftree}
                        s.t. $ Hp[\Pi] \subseteq \Gamma $ and $ Hp[\Pi] \vDash_2 \phi $.
                    \end{proofidea}
                \end{frame}

                \begin{frame}
                    \frametitle{Completeness}
                    \begin{itemize}
                        \item With the same assumptions as before, we can also prove \textbf{Completeness}
                    \end{itemize}
                    \begin{theorem}[Completeness]
                        $ \Gamma \vDash_2 \phi \Rightarrow \Gamma \vdash_2 \phi $
                    \end{theorem}
                    \begin{proofidea}
                        Immediate consequence of flattening SOL and the \textit{translation procedure} defined above, since $ \Gamma \vDash \phi \Rightarrow \Gamma \vdash \phi $ in FOL.
                    \end{proofidea}
                    \begin{itemize}
                        \item By combining the two previous results, we obtain
                    \end{itemize}
                    \begin{proposition}
                        $ \Gamma \vDash_2 \phi \Leftrightarrow \Gamma \vdash_2 \phi $
                    \end{proposition}
                \end{frame}

            \subsection{Identity}
                \begin{frame}
                    \frametitle{Identity (1)}
                    \begin{itemize}
                        \item The German philosopher \textbf{Gottfried Wilhelm Leibniz} introduced the \textit{Leibniz's law}, also called \textit{identity of indiscernibles} or \textit{Leibniz identity}
                    \end{itemize}
                    \begin{definition}[Leibniz identity]
                        Given two \textit{individuals} $ x $ and $ y $, we can state that $ x = y $ iff they have exactly the same properties.\\
                        Formally
                        \[
                            \forall X. (X(x) \Leftrightarrow X(y))    
                        \] 
                    \end{definition}
                    \begin{itemize}
                        \item Since we can now \textit{quantify} properties, SOL can easily represent this concept
                    \end{itemize}
                \end{frame}

                \begin{frame}
                    \frametitle{Identity (2)}
                    \begin{theorem}
                        $ \vdash_2 (x = y) \Leftrightarrow x = y $ w.r.t. \textit{Leibniz's law}
                    \end{theorem}
                    \begin{itemize}
                        \item To prove the above theorem more easily, we'll need the following properties
                    \end{itemize}
                    \begin{property}
                        \begin{enumerate}
                            \item[$ I_{\,1} $] $ \vdash_2 (x = x) $
                            \item[$ I_{\,2} $] $ \vdash_2 (x = y) \Rightarrow y = x $
                            \item[$ I_{\,3} $] $ \vdash_2 (x = y) \Rightarrow (y = z \Rightarrow x = z) $
                            \item[$ I_{\,4} $] $ \vdash_2 (x = y) \Rightarrow (\phi(x) \Rightarrow \phi(y)) $
                        \end{enumerate}
                    \end{property}
                \end{frame}

                \begin{frame}
                    \frametitle{Identity (3)}
                    \begin{proof}
                        \begin{itemize}
                            \item [$ \Rightarrow $] Immediately follows by $ I_{\,4} $.
                            \item [$ \Leftarrow $] We can derive $ (x = y) $ like this
                                \begin{prooftree}
                                    \AxiomC{$ x = x $}
                                    \AxiomC{$ \forall X. (X(x) \Leftrightarrow X(y)) $}
                                    \RightLabel{($ \forall^2 E $)}
                                    \UnaryInfC{$ x = x \Leftrightarrow x = y $}
                                    \RightLabel{($ \Rightarrow E $)}
                                    \BinaryInfC{$ x = y $}
                                \end{prooftree}
                        \end{itemize}
                    \end{proof}
                \end{frame}

            \subsection{Second-Order arithmetic}
                \begin{frame}
                    \frametitle{Second-Order arithmetic}
                    \begin{itemize}
                        \item Consider SOL with \textit{First-Order identity} and $ S $ predicate constant for the \textbf{successor relation}
                        \item We can add the following axioms:
                            \begin{enumerate}
                                \item $ \exists !x. \forall y. \neg S(y, x) $
                                \item $ \forall x. \exists !y. S(x, y) $
                                \item $ \forall x, y, z . (S(x, z) \wedge S(y, z) \Rightarrow x = y) $
                                \item $ \forall X.(X(0) \wedge \forall x. (X(x) \Rightarrow X(x^+)) \Rightarrow \forall X. X(x)) $
                            \end{enumerate}
                        \item The latter holds under the following axioms:
                            \begin{enumerate}[I]
                                \item $ \forall y. \neg S(y, 0) $
                                \item $ \forall \texttt{n}. S(\texttt{n}, \texttt{n + 1}) $
                                \item $ y = x^+ \Leftrightarrow S(x, y) $
                            \end{enumerate}
                    \end{itemize}
                \end{frame}

            \subsection{Axiom of Extensionality}
                \begin{frame}
                    \frametitle{Axiom of Extensionality}
                    \begin{itemize}
                        \item Let's define the concept of \textit{identity} for Second-Order terms
                    \end{itemize}
                    \begin{definition}
                        Given arguments $ t_1, \dots, t_n $, we can state that two \textit{predicates} $ X_1^n $ and $ X_2^n $ are \textit{identical} iff $ X_1^n(t_1, \dots, t_n) \Leftrightarrow X_2^n(t_1, \dots, t_n) $.\\
                        Formally
                        \[
                            \forall \{ t_1, \dots, t_n \}. (X_1^n(t_1, \dots, t_n) \Leftrightarrow X_2^n(t_1, \dots, t_n)) \Leftrightarrow X_1^n = X_2^n
                        \]
                    \end{definition}
                    \begin{itemize}
                        \item It allows \textit{definition by abstraction} (e.g. ``\textit{The set of all $ x $, such that...}")
                    \end{itemize}
                \end{frame}

                \begin{frame}
                    \vspace*{1cm}
                    \begin{center}
                        \begin{Huge}
                            \textcolor{title_blue}{Thank you for your attention!}
                        \end{Huge}
                    \end{center}
                \end{frame}
\end{document}
